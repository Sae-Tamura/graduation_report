\expandafter\ifx\csname ifdraft\endcsname\relax
    \documentclass[a4paper,12pt]{jreport}
    \usepackage[top=35truemm,bottom=30truemm,left=30truemm,right=30truemm]{geometry}
    \begin{document}
\fi


\chapter{関連研究}
ここに関連研究を紹介

\section{シーングラフ生成モデル}
本節では, シーングラフ生成モデルの歴史と代表的な手法について述べる.\\

シーングラフ生成(Scene Graph Generation: SGG)は, 与えられた画像からオブジェクトを検出し, それらのオブジェクト間の関係を予測することを目的とするタスクである. 各オブジェクトをノード, 関係をエッジとして表すことで, 画像の内容を構造的かつ解釈可能な形で記述できる点が特徴である. シーングラフ表現は, 画像キャプション生成やVisual Question Answering(VQA), 画像検索などの様々な下流タスクにおいて有用であることが示されている. 

\subsection{シーングラフ生成の歴史}
歴史書く

以下,代表的な手法を述べる

\subsection{EGTR}


\section{大規模言語モデル}
ちょい説明

\subsection{LLMの歴史}
歴史書く

\subsection{GTP}


\expandafter\ifx\csname ifdraft\endcsname\relax
    \end{document}
    \bibliography{main.bib}
\fi